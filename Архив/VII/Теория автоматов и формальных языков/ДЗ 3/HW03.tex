\documentclass[12pt,a4paper]{article}%
\usepackage{amsthm}
\usepackage{amsmath}%
\usepackage{amsfonts}%
\usepackage{amssymb}%
\usepackage{graphicx}
\usepackage[T2A]{fontenc}
\usepackage[utf8]{inputenc}
\usepackage[english,russian]{babel}
%-------------------------------------------
\setlength{\textwidth}{7.0in}
\setlength{\oddsidemargin}{-0.35in}
\setlength{\topmargin}{-0.5in}
\setlength{\textheight}{9.0in}
\setlength{\parindent}{0.3in}

\newtheorem{theorem}{Theorem}
\newtheorem{task}[theorem]{Задача}
\addto\captionsrussian{\renewcommand*{\proofname}{Решение}}


\newcommand{\abovemath}[2]{\ensuremath{\stackrel{\text{#1}}{#2}}}
\newcommand{\aboveeq}[1]{\abovemath{#1}{=}}
\newcommand\bydef{\aboveeq{def}}
\begin{document}


\begin{flushright}
\textbf{Константин Киреев 8383 \\
\today}
\end{flushright}

\begin{center}
\textbf{Формальные языки\\
HW03 \\
Дедлайн: 23:59 15 ноября 2021} \\
\end{center}

\task{
    Привести однозначную контекстно-свободную грамматику для языка арифметических выражений над положительными целыми \emph{числами} с операциями \verb!+!, \verb!-!, \verb!*!, \verb!/!, \verb!^!, \verb!==!,\verb!<>!, \verb!<!, \verb!<=!, \verb!>!, \verb!>=! со следующими приоритетами и ассоциативностью:

  \begin{center}
    \begin{tabular}{ c | c }
      Наибольший приоритет & Ассоциативность  \\ \hline \hline
      \verb!^! & Правоассоциативна \\
     \verb!*!,\verb!/! & Левоассоциативна \\
     \verb!+!,\verb!-! & Левоассоциативна \\
     \verb!==!,\verb!<>!, \verb!<!,\verb!<=!, \verb!>!,\verb!>=! & Неассоциативна \\ \hline \hline
     Наименьший приоритет & Ассоциативность
    \end{tabular}
    \end{center}


    Неассоциативные операции встречаются только один раз: \verb!1 == 2! -- корректная строка, \verb!1 == 2 == 3!, \verb!(1 == 2) == 3!, \verb!1 < 2 > 3! --- некорректные строки
  }

\begin{proof}
  
  \begin{align}
    G=<\{N,P,K,R,S\},\{0..9\},P,S>
  \end{align}

  \begin{center}
    \begin{align*}
      S  &\to P==P \mid P\neq P \mid P<P \mid P \leq P \mid P > P \mid P \geq P \\
      P  &\to P+K \mid P-K \mid K \\
      K  &\to K \cdot R \mid K/R \mid R \\
      R  &\to R\string^N \mid N \\
      N  &\to 0\mid ..\mid 9
    \end{align*}
  \end{center}

\end{proof}

\task {Привести грамматику из 1 задания в нормальную форму Хомского.}

\begin{proof}

    \begin{align*}
      S  &\to C_0 P \mid C_1 P \mid C_2 P \mid C_3 P \mid C_4 P \mid C_5 P \\
      P  &\to C_6 K \mid C_7 K \mid C_8 R \mid C_9 R \mid C_{10} N \mid 0 \mid 1 \mid 2 \mid 3 \mid 4 \mid 5 \mid 6 \mid 7 \mid 8 \mid 9 \\
      K  &\to C_8 R \mid C_9 R \mid C_{10} N \mid 0 \mid 1 \mid 2 \mid 3 \mid 4 \mid 5 \mid 6 \mid 7 \mid 8 \mid 9 \\
      R  &\to C_{10} N \mid 0 \mid 1 \mid 2 \mid 3 \mid 4 \mid 5 \mid 6 \mid 7 \mid 8 \mid 9 \\
      N  &\to 0 \mid 1 \mid 2 \mid 3 \mid 4 \mid 5 \mid 6 \mid 7 \mid 8 \mid 9 \\
      C_0  &\to P C_{11} \\
      C_1  &\to P C_{12} \\
      C_2  &\to P C_{13} \\
      C_3  &\to P C_{14} \\
      C_4  &\to P C_{15} \\
      C_5  &\to P C_{16} \\
      C_6  &\to P C_{17} \\
      C_7  &\to P C_{18} \\
      C_8  &\to K C_{19} \\
      C_9  &\to K C_{20} \\
      C_{10}  &\to R C_{21} \\
      C_{11}  &\to == \\
      C_{12}  &\to \neq \\ 
      C_{13}  &\to < \\
      C_{14}  &\to \leq \\`'
      C_{15}  &\to > \\ 
      C_{16} &\to \geq \\
      C_{17}  &\to + \\
      C_{18}  &\to - \\
      C_{19}  &\to * \\
      C_{20}  &\to / \\
      C_{21}  &\to \string^
    \end{align*}

\end{proof}


\task {Промоделировать работу алгоритма CYK на грамматике из 2 задания на трех корректных строках не короче 7 символов и на трех некорректных строках. (Привести таблицы и деревья вывода)}

\begin{proof}
  -
\end{proof}

\end{document}