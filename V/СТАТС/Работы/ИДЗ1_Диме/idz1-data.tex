\problemset{Статистический анализ}
\problemset{Индивидуальное домашнее задание №1}

\renewcommand*{\proofname}{Решение}

Плотность двумерного нормального распределения имеет вид:
\[ p_{\xi,\eta}(x, y) = C \cdot \exp \left\{-\frac12(3x^2 - 4xy + 6y^2 - 2x - 8y + 5)\right\} \]

%% Задание 1: условие
\begin{problem}
	Вычислить вектор мат. ожиданий и ковариационные характеристики данного случайного вектора.
\end{problem}

%% Задание 1: решение
\begin{proof}
	Обозначим степень экспоненты через  $q(x,y)$. Тогда:
	\begin{align} 
		q(x,y)& = (3x^2-4xy-2x)+6y^2-8y+5= \\
		& = ((\sqrt{3x})^2-2\cdot \sqrt{3}x(y\frac{2}{\sqrt{3}}+\frac{1}{\sqrt{3}})+(\frac43y^2+\frac43y+\frac13))+6y^2-\frac43y^2-8y+5-\frac43y-\frac13= \\
		&  = (\sqrt{3}x-\frac{2}{\sqrt{3}}y-\frac{1}{\sqrt{3}})^2+\frac{14}{3}y^2-\frac{28}{3}y+\frac{14}{3}=(\sqrt{3}x-\frac{2\sqrt{3}}{3}y-\frac{\sqrt{3}}{3})^2+\frac{14}{3}(y^2-2y+1)= \\
		& = \underline{(\sqrt{3}x-\frac{2\sqrt{3}}{3}y-\frac{\sqrt{3}}{3})^2+\frac{14}{3}(y-1)^2}=([\sqrt{3}x-\sqrt{3}]-\frac{2\sqrt{3}}{3}[y-1])^2+\frac{14}{3}(y-1)^2= \\
		& = 3(x-1)^2-4(x-1)(y-1)+\frac43(y-1)^2+\frac{14}{3}(y-1)^2 = \\
		& = 3(x-1)^2+2\cdot(-2)\cdot(x-1)(y-1)+6(y-1)^2
	\end{align}

	Получаем:	
	$ \mathbb E\begin{pmatrix} \xi \\ \eta \end{pmatrix} = \begin{pmatrix} 1 \\ 1 \end{pmatrix}$,
	
	$ \Sigma^{-1} = \begin{pmatrix} 3 & -2 \\ -2 & 6 \end{pmatrix} $, 
	$ \Sigma = \cfrac1{14}\begin{pmatrix} 6 & 2 \\ 2 & 3 \end{pmatrix} $,
	
	$  \cov(\xi, \xi)=\mathbb D\xi = \sigma^2_\xi = \cfrac6{14} = \cfrac37 $,
	
	$  \cov(\eta, \eta)=\mathbb D\eta = \sigma^2_\eta = \cfrac3{14} $,
	
	$ \cov(\xi, \eta) = \cfrac17 $,
	$ \rho_{\xi, \eta}= \cfrac{\cov(\xi, \eta)}{\sqrt{D_\xi D_\eta}}=\cfrac{\frac17}{\frac{\sqrt{9}}{98}} = \cfrac{\sqrt2}3 $,
	
	$ C = \cfrac1{(2\pi)^{n/2}\cdot\sqrt{\det\Sigma}} = \cfrac{\sqrt{14}}{2\pi} $.
	
\end{proof}
%\newpage
%% Задание 2: условие
\begin{problem}
	Найти аффинное преобразование, переводящее исходный случайный вектор в стандартный нормальный. 
\end{problem}

%% Задание 2: решение
\begin{proof}	
	\begin{multline} \begin{pmatrix} \zeta_1 \\ \zeta_2 \end{pmatrix}=\begin{pmatrix} \sqrt3 & -\frac{2\sqrt{3}}{3} \\ 0 & \frac{\sqrt{42}}{3} \end{pmatrix}\begin{pmatrix} \xi_1-1 \\ \xi_2-1 \end{pmatrix}=\begin{pmatrix} \sqrt{3}\xi_1-\frac{2\sqrt{3}}{3}\xi_2-\frac{\sqrt{3}}{3} \\ \frac{\sqrt{42}}{3}\xi_2-\frac{\sqrt{42}}{3} \end{pmatrix};	
	\\B \Sigma B^T = I \Rightarrow 	
	\cfrac1{14}\begin{pmatrix} \sqrt3 & -\frac{2\sqrt{3}}{3} \\ 0 & \frac{\sqrt{42}}{3} \end{pmatrix}\begin{pmatrix} 6 & 2 \\ 2 & 3 \end{pmatrix}\begin{pmatrix} \sqrt3 & 0 \\ -\frac{2\sqrt{3}}{3} & \frac{\sqrt{42}}{3} \end{pmatrix}=\cfrac1{14}\begin{pmatrix} \frac{14\sqrt{3}}{3} & 0 \\ \frac{2\sqrt{42}}{3} & \sqrt{42} \end{pmatrix}\begin{pmatrix} \sqrt3 & 0 \\ -\frac{2\sqrt{3}}{3} & \frac{\sqrt{42}}{3} \end{pmatrix}=\\=\cfrac1{14}\begin{pmatrix} 14 & 0 \\ 0 & 14 \end{pmatrix}=\begin{pmatrix} 1 & 0 \\ 0 & 1 \end{pmatrix}\Rightarrow \begin{pmatrix} \zeta_1 \\ \zeta_2 \end{pmatrix}\sim N(0, I).	
	\end{multline} 		
\end{proof}


%% Задание 3: условие
\begin{problem}
	Найти ортогональное преобразование, переводящее соответствующий центрированный случайный вектор в вектор с независимыми компонентами. 
\end{problem}

%% Задание 3: решение
\begin{proof}
	$ $		
		
	$ \Sigma^{-1}=\begin{pmatrix} 3 && -2 \\ -2 && 6 \end{pmatrix} $
		
	$ \det\begin{pmatrix}\Sigma^{-1}-\lambda I\end{pmatrix} = 0$
	
	$\det\begin{pmatrix} 3-\lambda && -2 \\ -2 && 6-\lambda \end{pmatrix}=0\Rightarrow	(3-\lambda)(6-\lambda)-4=0; 18-3\lambda-6\lambda+\lambda^2-4=0; $
	
	$ \lambda^2-9\lambda+14=0 \Rightarrow \underline{\lambda_1=2}; \underline{\lambda_2=7} $
 	
 	$ \lambda_1=2: \begin{pmatrix} 1 && -2 \\ -2 && 4\end{pmatrix}\Leftrightarrow\begin{pmatrix} 1 && -2 \end{pmatrix}; x=\begin{pmatrix} 2 \\ 1\end{pmatrix}\cfrac{1}{\sqrt{5}} $	
 	
 	$ \lambda_2=7: \begin{pmatrix} -4 && -2 \\ -2 && -1\end{pmatrix}\Leftrightarrow\begin{pmatrix} 2 && 1 \end{pmatrix}; x=\begin{pmatrix} 1 \\ -2\end{pmatrix}\cfrac{1}{\sqrt{5}} $	
 	
 	$ \underline{Q=\cfrac{1}{\sqrt{5}}\begin{pmatrix} 2 && 1 \\ 1 && -2\end{pmatrix}} - \text{матрица ортогональных преобразований} $
 	
 	$ \mathbb E\begin{pmatrix} \zeta_1 \\ \zeta_2 \end{pmatrix}=Q\cdot\mathbb E\begin{pmatrix} \xi \\ \eta \end{pmatrix}=Q\begin{pmatrix} 1 \\ 1 \end{pmatrix}=\cfrac{1}{\sqrt{5}}\begin{pmatrix} 2 && 1 \\ 1 && -2 \end{pmatrix}\begin{pmatrix} 1 \\ 1 \end{pmatrix} = \cfrac{1}{\sqrt{5}}\begin{pmatrix} 3 \\ -1 \end{pmatrix} $
 	
 	$ Q \Sigma Q^T; \Sigma = \cfrac1{14}\begin{pmatrix} 6 & 2 \\ 2 & 3 \end{pmatrix}; $
 	
 	$ Q \Sigma Q^T = \cfrac{1}{\sqrt{5}}\begin{pmatrix} 2 & 1 \\ 1 & -2 \end{pmatrix}\cdot\cfrac{1}{14}\begin{pmatrix} 6 & 2 \\ 2 & 3 \end{pmatrix}\cdot\cfrac{1}{\sqrt{5}}\begin{pmatrix} 2 & 1 \\ 1 & -2 \end{pmatrix}=\cfrac15\cdot\cfrac1{14}\begin{pmatrix} 14 & 7 \\ 2 & -4 \end{pmatrix}\begin{pmatrix} 2 & 1 \\ 1 & -2 \end{pmatrix}= $
 	
 	$ =\cfrac15\cdot\cfrac1{14}\begin{pmatrix} 35 & 0 \\ 0 & 10 \end{pmatrix} $

	$ \begin{pmatrix} \zeta_1 \\ \zeta_2 \end{pmatrix} \rightsquigarrow M \left(\begin{pmatrix} 3/\sqrt{5} \\ -1/\sqrt{5} \end{pmatrix}, \cfrac1{14}\begin{pmatrix} 7 & 0 \\ 0 & 2 \end{pmatrix}\right) $
	
\end{proof}

%% Задание 4: условие
\begin{problem}
	Вычислить характеристики совместного распределения случайного вектора \\ (3$\xi$ - 2$\eta$, -2$\xi$ - 2$\eta$) и записать его плотность. 
\end{problem}

%% Задание 4: решение
\begin{proof}
	$ $
	
	$ \begin{pmatrix} \zeta_1 \\ \zeta_2 \end{pmatrix}=\begin{pmatrix} 3\xi-2\eta \\ -2\xi-2\eta \end{pmatrix} $
	
	$ x \sim N(\mu, \Sigma); Y=BX; Y \sim N(B\mu, B\Sigma B^T); $
	
	$ \mathbb E\begin{pmatrix} \xi \\ \eta \end{pmatrix} = \begin{pmatrix} 1 \\ 1 \end{pmatrix}; \Sigma = \cfrac1{14}
	\begin{pmatrix} 6 && 2 \\ 2 && 3 \end{pmatrix}; $ 	

	$ \mathbb E\begin{bmatrix} 3\xi-2\eta \\ -2\xi-2\eta \end{bmatrix}=\begin{bmatrix} 3 && -2 \\ -2 && -2 \end{bmatrix}\begin{bmatrix} 1 \\ 1 \end{bmatrix}=\begin{pmatrix} 1 \\ -4 \end{pmatrix}$	
	
	$ \Sigma_Y=B \Sigma B^T=\cfrac1{14}\begin{bmatrix} 3 && -2 \\ -2 && -2 \end{bmatrix}\begin{bmatrix} 6 && 2 \\ 2 && 3 \end{bmatrix}\begin{bmatrix} 3 && -2 \\ -2 && -2 \end{bmatrix}=\cfrac1{14}\begin{bmatrix} 14 && 0 \\ -16 && -10 \end{bmatrix}\begin{bmatrix} 3 && -2 \\ -2 && -2 \end{bmatrix}=$ 	
	
	$ =\cfrac1{14}\begin{bmatrix} 42 && -28 \\ -28 && 52 \end{bmatrix} $
	
	$ \det(\Sigma_Y)=\cfrac{50}{7} $
	
	$ \Sigma_Y^{-1}=\cfrac1{100}\begin{pmatrix} 52 && 28 \\ 28 && 42 \end{pmatrix}$
	
	$ p_{\zeta_1,\zeta_2}(x, y) = \cfrac1{2\pi\sqrt{\frac{50}{7}}} \cdot \exp \left\{-\cfrac12\begin{pmatrix} x-1 \\ y+4 \end{pmatrix}^T\cdot\cfrac1{100}\begin{pmatrix} 52 && 28 \\ 28 && 42 \end{pmatrix}\cdot\begin{pmatrix} x-1 \\ y+4 \end{pmatrix}\right\} $
	
	Обозначим степень экспоненты как $\left(-\frac12\right)\cdot q(x, y)$. Тогда:
	\begin{align}
	q(x, y)& =\left(\cfrac1{100}\right)\cdot\begin{pmatrix} x-1 && y+4 \end{pmatrix}\begin{pmatrix} 52 && 28 \\ 28 && 42 \end{pmatrix}\begin{pmatrix} x-1 \\ y+4 \end{pmatrix}= \\	
	& =\left(\cfrac1{100}\right)\cdot\begin{pmatrix} 52(x-1)+28(y+4) && 28(x-1)+42(y+4) \end{pmatrix}\begin{pmatrix} x-1 \\ y+4 \end{pmatrix}= \\	
	& =\left(\cfrac1{100}\right)\cdot\begin{pmatrix} 52(x-1)^2+56(x-1)(y+4)+42(y+4)^2\end{pmatrix}= \\	
	& =\left(\cfrac1{100}\right)\cdot(52x^2+56xy+42y^2+120x+280y+500)= \\	
	& =\cfrac{13}{25}x^2+\cfrac{14}{25}xy+\cfrac{21}{50}y^2+\cfrac65x+\cfrac{14}5y+5 \Rightarrow
	\end{align}
	
	\begin{equation}
		\underline{  
		p_{\zeta_1,\zeta_2}(x, y) = \cfrac1{2\pi\sqrt{\frac{50}{7}}} \cdot \exp \left\{-\cfrac12\cdot\left(\cfrac{13}{25}x^2+\cfrac{14}{25}xy+\cfrac{21}{50}y^2+\cfrac65x+\cfrac{14}5y+5\right)\right\}}
	\end{equation}  
\end{proof}

%% Задание 5: условие
\begin{problem}      
	Найти условное распределение $\xi$ при условии $\eta$. 
\end{problem}

%% Задание 5: решение
\begin{proof}	
	$ $
	
	\large $ p_{\xi,\eta}(x, y) = C \cdot \exp \left\{-\frac12(3x^2 - 4xy + 6y^2 - 2x - 8y + 5)\right\} $
	
	$ p_{\xi|\eta=y}=\cfrac{p_{\xi,\eta}(x, y)}{p_{\eta}(y)}= \cfrac{C \cdot \exp \left\{-\frac12(3x^2 - 4xy + 6y^2 - 2x - 8y + 5)\right\}}{C_1(y)}= $
	
	$ =C_2(y) \cdot \exp \{-\frac12(3x^2-2x-4xy)\underbrace{-\frac12(6y^2-8y+5)}_{C_3(y)}\}= $
	
	\begin{multline}	
	=C_2(y) \cdot \exp \Bigg\{-\cfrac12\cdot\cfrac1{1/3}\bigg(x^2-2\cdot x\cdot\bigg(\cfrac23y+\cfrac13\bigg)+\bigg(\cfrac23 y+\cfrac13\bigg)^2\bigg)+\\\underbrace{+\cfrac1{2\cdot 1/3}\cdot\bigg(\cfrac23y+\cfrac13\bigg)^2 -\frac12\bigg(6y^2-8y+5\bigg)}_{C_3(y)}\Bigg\}=
	\end{multline}

	\large $ =C_4(y) \cdot \exp \left\{\cfrac{-\bigg(x-\bigg(\cfrac23 y+\cfrac13\bigg)\bigg)^2}{2\cdot\bigg(\cfrac{\sqrt{3}}{3}\bigg)^2}\right\}; $ \Large $ \mu=\frac23y+\frac13; \sigma=\frac{\sqrt{3}}{3}; $ \large
	
	$ \mathbb E(\xi|\eta=y)=\cfrac23\eta+\cfrac13; $ 	
	$ \mathbb D(\xi|\eta=y)=\cfrac13; $ 	
	$ \sigma=\cfrac{\sqrt{3}}{3}; $ 
	
	$ C = \cfrac1{\sqrt{2\pi\sigma^2}}=\cfrac1{\sqrt{2\pi\frac13}}=\sqrt{\cfrac3{2\pi}}; $
	
	$ p_{\xi|\eta=y}(x, y)=\sqrt{\cfrac3{2\pi}}\cdot \exp \left\{\cfrac{-\bigg(x-\bigg(\cfrac23 y+\cfrac13\bigg)\bigg)^2}{2\cdot\bigg(\cfrac{\sqrt{3}}{3}\bigg)^2}\right\}; $
	
\end{proof}







